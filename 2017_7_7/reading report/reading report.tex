\documentclass[UTF8]{article}

\usepackage{geometry}

\special{papersize=2in,2in}%纸张大小为8.5inch×11inch

\usepackage{amsmath,amssymb}  % 数学字符与公式宏包,分数多用\dfrac方能显示我认可的大小
\usepackage{calc}
\usepackage{units} %单位宏包
\usepackage{graphicx}   % 插图必用宏包
\usepackage{enumerate}



% opening

\begin{document}


	\title{Journal club report}	
	\date{July 17, 2017} 
	\maketitle
    	We discussed “Tensorization and structured tensors”\\
    	The main idea of this paper:
    	
    	This chapter propose several tensorization methods, which mainly obtained through three aspects:
    		\begin{enumerate}[1)]
    		\item Rearrangement (reshape).
    		\item Alignment of data blocks/epochs.
    		\item Data augmentation(Toeplitz and Hankel tensors).
    	\end{enumerate}
  
    	Main discussion:
    	\begin{enumerate}[1)]
    		\item The definition and construction of Toeplitz/Hankel tensor, Convolution Tensor.
    		\item	Longxi represent the blind source separation in details.
    		\item	Image denoising by learning local structures, main idea is generating tensors form local blocks(patches) which are similar or closely related.
    	\end{enumerate}
    
   	
    		

\end{document}