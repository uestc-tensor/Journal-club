\documentclass[UTF8]{article}

\usepackage{geometry}

\special{papersize=2in,2in}%纸张大小为8.5inch×11inch

\usepackage{amsmath,amssymb}  % 数学字符与公式宏包,分数多用\dfrac方能显示我认可的大小
\usepackage{calc}
\usepackage{units} %单位宏包
\usepackage{graphicx}   % 插图必用宏包
\usepackage{enumerate}



% opening

\begin{document}


	\title{Journal club report}	
	\date{Nov 3, 2017} 
	\maketitle
	The main idea of these paper:
	
	MPS(matrix product state) algorithm: applying MPS on training set to obtain the common  factors $\mathcal{B}_{i_1}^{(1)},\cdots,\mathcal{B}_{i_{n-1}}^{(n-1)}  ,\mathcal{C}_{i_n}^{(n)},\cdots,\mathcal{C}_{i_N}^{(N)}$  and core matrix $\mathcal{G}_k^{(n)}$,then use the common factors the obtain the core matrix of the test set.
	
	For the Tmac+TT, due to the blocking-artifacts, introducing an operation called image concatenation.
	
	Main discussion:
	
	MPS (matrix product state) algorithm:  
	\begin{enumerate}[1)]
	\item the mode of MPS and the mode of tensor train is similar; 
	\item using the feature matrix obtained through this algorithm to classification, the result is very good. Because the common factors are very detailed.
     \end{enumerate}
 
	Classification: we discuss the classification process, the conclusion is that this paper just uses simple classification algorithm. It does’t have a feedback to update the common factors or improve the quality.
	
	image concatenation operation: we think this operation is analogous to the shift operation we used in block multi-low rank decomposition to reduce the blocking-artifacts. The theory of this operation is unknown.
	
	The possible work: 1) KA+MPS; 2) the application for MPS.

 
    		

\end{document}